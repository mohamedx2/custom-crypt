\documentclass[a4paper, 12pt]{article}
\usepackage{graphicx}
\usepackage{amsmath}
\usepackage{hyperref}

\title{Custom Crypt: A Secure Cryptography Project}
\author{Mohamed Ali Hamroun \texttt{(\url{http://mohamedalihamroun.me})}}
\date{\today}

\begin{document}

\maketitle

\begin{abstract}
This report provides an overview of the Custom Crypt project, which aims to implement secure cryptographic algorithms. The project focuses on ensuring data confidentiality and integrity through various encryption techniques.
\end{abstract}

\section{Introduction}
In today's digital age, data security is of paramount importance. The Custom Crypt project addresses the need for secure data transmission and storage by implementing robust cryptographic methods.

\section{Project Overview}
The Custom Crypt project includes various cryptographic algorithms, such as symmetric and asymmetric encryption. The main goal is to provide a user-friendly interface for encrypting and decrypting data securely.

\section{Implementation Details}
The implementation of the Custom Crypt project is primarily done in Python. The key components include:
\begin{itemize}
    \item \textbf{Encryption Algorithms:} Description of the algorithms used (e.g., AES, RSA).
    \item \textbf{Key Management:} How keys are generated, stored, and managed.
    \item \textbf{User Interface:} Overview of the command-line interface or graphical user interface.
\end{itemize}

\section{Hashing Method}
The Custom Crypt project implements a custom hashing function named \texttt{manual\_hash}. This function is designed to provide a unique hash for input strings, utilizing an optional salt for added uniqueness.

\subsection{Function Definition}
The function is defined as follows:

\begin{verbatim}
def manual_hash(input_string: str, salt: str = "") -> int:
\end{verbatim}

\subsection{Parameters}
\begin{itemize}
    \item \textbf{input\_string:} The string that you want to hash.
    \item \textbf{salt:} An optional string that can be added to the input for uniqueness.
\end{itemize}

\subsection{Return Value}
The function returns a 64-bit integer as the resulting hash.

\subsection{Implementation Details}
The \texttt{manual\_hash} function operates as follows:
\begin{itemize}
    \item It concatenates the \texttt{salt} with the \texttt{input\_string}.
    \item The hash value is initialized with a large prime number: \texttt{0xABCDEF1234567890}.
    \item The function iterates over each character in the combined string, using the ASCII value of each character to mix into the hash value through bitwise operations.
\end{itemize}

\subsection{Example Usage}
The function can be tested with a sample input, such as:
\begin{verbatim}
if __name__ == "__main__":
    test_string = "hello world"
    print("Hash with default salt:")
\end{verbatim}

\section{Conclusion}
The Custom Crypt project successfully implements secure cryptographic techniques to protect sensitive data. Future work may include enhancing the user interface and adding more algorithms.

\section{References}
\begin{thebibliography}{9}
\bibitem{crypto}
    Cryptography and Network Security: Principles and Practice, William Stallings.
\end{thebibliography}

\end{document}
